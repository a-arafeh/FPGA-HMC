\documentclass[11pt,twocolumn]{article}

\usepackage{graphicx}  % Required to insert images
\usepackage{amsmath}   % Required for math
\usepackage{subfigure} % Required to insert sub-images
\usepackage{mathptmx}  % Required to use Times Roman font
\usepackage{geometry}  % Required to set margins
 \geometry{
 letterpaper, %a4paper
 total={215.9mm, 279.4mm}, %210mm,297mm
 left=15mm,
 right=15mm,
 top=10mm,
 bottom=20mm,
 }

\begin{document}

\title{\textbf{Design Space Exploration of New FPGAs-HMCs System Architecture for Graph Applications}}

\author{\large Abdalrahman M. Arafeh and Guy G.F. Lemieux \\
Department of Electrical and Computer Engineering\\
University of British Columbia \\ Vancouver, B.C., V6T 1Z4, Canada \\
\{arafeh, lemieux\}@ece.ubc.ca}

\date{\vspace{-4ex}}


\maketitle

Algorithms that exhibit irregular memory access patterns are known
to show poor performance on multiprocessor architectures,
particularly when those applications have low computation to
memory access ratio. Graph-based algorithms exhibit very poor
temporal and spacial locality and they do not benefit from
conventional latency reduction in cash-based systems. We are
aiming to make an early design space exploration of a new
multi-core multi-threaded FPGA-based system architecture for high
performance computing of irregular graph applications. We are
using FPGAs for their flexibility and for their serial
interconnects capability and interpretability with fast memory
systems like the Hybrid Memory Cube (HMC).

The main challenge for speeding-up graph applications is memory
access concurrency. System performance and scalability will be
influenced by the unpredictable memory access. While
multi-threaded processor cores can hide memory latency, the
overall system performance will be limited by available memory
banks and the number of simultaneous memory reads and writes.
Current DDR3 SDRAM memories are limited to 8 banks and they are
not designed for random memory accesses. Whereas, HMC can handle
out-of-order memory requests through operationally independent
vaults controllers, packets queueing, many memory banks and
high-speed serial links. HMC requires smaller board space, less
routing, physical volume and logic for memory controllers compared
to DDR3 memories with equivalent number of banks.

In envisioning a new system architecture we can model the system
as to have multiple FPGAs each with $C$ cores and $T$ thread
contexts in one of the topologies shown in Figure~\ref{f1}. Such
architecture offers many design choices including different
FPGA-to-HMC ratios and topologies trade-offs (4-to-1 in
Figure~\ref{1-1-1}, 1-to-8 in Figure~\ref{1-1-3}, 2-to-8 in
Figure~\ref{1-1-4} and 4-to-8 in Figure~\ref{1-1-5}). Furthermore,
it provides the chance to use FPGAs with different sizes and
different possible FPGA-to-FPGA connectivity. The architecture
will also provide a hardware infrastructure for different
programming models including High Level Synthesis, soft processors
and overlays.

To simulate the new architecture we are using
HMC-SIM~\cite{HMCSIM}; a basic HMC simulator made in Texas Tech
University. HMC-SIM simulates the internal HMC crossbar's and
vaults queues and detects bank conflicts, but it lacks timing
information, transmission links modelling and memory access
latency. Currently, we are limited by HMC-SIM capabilities and we
are looking for more detailed and accurate HMC simulator to
combine it with our architectural level simulations.

Graphs can integrate data across different sources and connect
location information, health conditions and driving habits to find
answers to questions like, how to retain customers, catch frauds,
or foresight threats and possible epidemics. Graph Analytics can
uncover valuable insights in Big Data. They are based on versatile
and powerful model that relates objects to objects with arbitrary
relationships and they can explain different phenomenons in the
world.

Graph Analytics and pointer-based data structure algorithms can
spawn concurrent activity for each element (e.g., graph node). Any
element can point to other elements leading to unpredictable,
irregular and random fine-grained memory accesses in datasets that
could be in the range of terabytes. Alternatively, tolerating
graph algorithms irregular memory latency can be achieved by
utilizing novel memory systems and by using multi-threaded
processors. Where, a large number of thread contexts can keep the
processor pipeline busy and new memory systems can provide
sustained and simultaneous memory operations.



\begin{figure*}[p]
\centering \subfigure[Topology-1: 4-FPGAs, 1-HMC - Max: 1 hop]{
\includegraphics[scale=0.6]{f1.eps}
\label{1-1-1} } \label{f1}
\end{figure*}

\begin{figure*}[p]
\centering \subfigure[Topology-2: 1-FPGA, 8-HMCs - Max: 3 hops]{
\includegraphics[scale=0.6]{f2.eps}
\label{1-1-3} }  \hspace{3em} \subfigure[Topology-3: 2-FPGAs,
8-HMCs - Max: 3 hops]{
\includegraphics[scale=0.6]{f3.eps}
\label{1-1-4} } \label{f1}
\end{figure*}


\begin{figure*}[p]
\centering \subfigure[Topology-4: 4-FPGAs, 8-HMCs - Max: 4 hops]{
\includegraphics[scale=0.6]{f6.eps}
\label{1-1-5} } \caption[]{FPGAs-HMCs System Architecture.}
\label{f1}
\end{figure*}

\bibliography{Irregular}
\bibliographystyle{unsrt}

\end{document}
